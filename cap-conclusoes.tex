%% ------------------------------------------------------------------------- %%
\chapter{Conclusões}
\label{cap:conclusoes}

Texto texto texto texto texto texto texto texto texto texto texto texto texto
texto texto texto texto texto texto texto texto texto texto texto texto texto
texto texto texto texto texto texto\footnote{Exemplo de referência para página
Web: \url{www.vision.ime.usp.br/~jmena/stuff/tese-exemplo}}.

%------------------------------------------------------
\section{Considerações Finais} 

Texto texto texto texto texto texto texto texto texto texto texto texto texto
texto texto texto texto texto texto texto texto texto texto texto texto texto
texto texto texto texto texto texto. 

%------------------------------------------------------
\section{Sugestões para Pesquisas Futuras} 

Texto texto texto texto texto texto texto texto texto texto texto texto texto
texto texto texto texto texto texto texto texto texto texto texto texto texto
texto texto texto texto texto texto.

Finalmente, leia o trabalho de Uri Alon \cite{alon09:how} no qual apresenta-se
uma reflexão sobre a utilização da Lei de Pareto para tentar definir/escolher
problemas para as diferentes fases da vida acadêmica.  A direção dos novos
passos para a continuidade da vida acadêmica deveriam ser discutidos com seu
orientador.
